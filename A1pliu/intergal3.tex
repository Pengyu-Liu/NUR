\section{Satellite galaxies around a massive central}
In this question, we need to write a numerical integrator to solve for A and interpolate the function. In the end, we need to write a function that can return the Poisson probability distribution for a given positive mean $\lambda$ and integer k.

The shared code is given by:
\lstinputlisting[firstline=1,lastline=5]{3integral.py}

\subsection{a}
This a 3D integral, but we can reduce it to a 1D integral. In this case, dV=4$\pi$x$^2$dx. The integral becomes $4\pi b^{3-a}\int_{0}^{5} (x^{a-1}exp(-(\frac{x}{b})^c) \,dx $. Because the integral does not contain any singularities from 0 to 5 at a given a=2.2, b=0.5 and c=3.1, we wrote an extended Simpson integrator to solve the integral. The we can calculate A=$\frac{1}{integral}$. The algorithm of extended Simpson's rule is given on slide7 in lecture3.\\

The script of this subquestion is given by:
\lstinputlisting[firstline=6,lastline=43]{3integral.py}

We set the number of intervals between 0 and 5 to be 1000. A is given by:
\lstinputlisting{A.txt}

\subsection{b}
Because we only have 5 points that span in a large range, I chose to do interpolation in loglog space. I used the linear interpolator because there are only 5 points and they show piecewise patterns: changes relatively slowly when x\textless 1 while changes dramatically when x\textgreater 1.

The code is given by:
\lstinputlisting[firstline=46,lastline=121]{3integral.py}

Figure 2 shows the five points and the results of linear interpolation.

\begin{figure}[h!]
  \centering
  \includegraphics[width=0.9\linewidth]{./plots/3binterp.png}
  \caption{The loglog plot of interpolation. I chose to do linear interpolation at loglog space and truncate at x=5. Although the curve of linear interpolation is not smooth, it does not derivate far from the five points and does not cause large wrinkles between points. It also follows the overall trend of these points.}
  \label{fig2}
\end{figure}


\subsection{c}
The Poisson probability distribution $P_\lambda(k)=\frac{\lambda^k e^{-\lambda}}{k!} $ is easy to underflow because its denominator k! can be very large and overflows even in float64 type. The numerator $\lambda^k$ can also be very large and exp(-$\lambda$) can be very small. Because we calculate the three parts separately in the computer, any part overflows or underflows will cause a huge relative error to P, though P itself is not easy to underflow. So I chose to calculate P at log space and return it back to linear space.

The code is given by:
\lstinputlisting[firstline=123,lastline=143]{3integral.py}

The results of given ($\lambda$,k) are given by:
\lstinputlisting{3c.txt}

We can see that this log function can return $P_\lambda(k)$ correctly in a large range by using numpy float64 type, which is much better than calculating P straight from its definition.
