\section{Cooling rates in cosmological simulations}

In this section we look at question Cooling rates in cosmological simulations,
we want to do the 3d interpolation to the cooling\_rates over Hydrogen density, temperature and redshift.
The 3D linear interpolator is given by the function linear\_interp3d(cube,x,y,z,xitp,yitp,zitp) in the shared modules.
For a given new point(znew,ynew,xnew), we use the bisection method to find the positions where we need to do interpolation for each axis.
After finding the nearest values for znew(z1,z2), ynew(y1,y2) and xnew(x1,x2), we do the linear interpolation along each axis.
For example, the xnew locates between x1 and x2, then we fix z1,y1 to do 1D linear interpolation along x axis.
Then we fix z1,y2 and also do 1D interpolation along x axis. And we use the two interpolated new points to do 1D linear interpolation along y axis.
Repeat the steps for z2, and we interpolate the final two points along the z axis.\\

The shared modules for the sub-question a and b are given by:
\lstinputlisting[firstline=3, lastline=97]{cooling.py}

\subsection{a}
We use the equation (2) in Hand in Exercise 1 to calculate the cooling rate. The cooling rate for H and He is obtained from the metal\_free file.
Because the coefficient for the second term on the right hand side of equation (2) is the same for every elements,
we can use the cooling rate in the Total\_metal file as the sum of cooling rate of all heavy elements. The metallicity for this question is 0.25.
So we need to do 3D linear interpolation for cooling rate(H,He), cooling rate(total metal), ne/nH and (ne/nH)solar separately.
Though (ne/nH)solar is the same for all redshifts, we still copy them into a 3D cube so that we can use the 3D linear interpolator.
Because electron density contributed from heavy elements is very small, we simply use the (ne/nH) in the metal\_free file.\\

The code specific to this question is given by:
\lstinputlisting[firstline=100, lastline=133]{cooling.py}

Setting metallicity=0.25 and at at a redshift of z = 3, our script produces the following results for densities of (1 cm$^{-3}$, 10$^{-2}$ cm$^{-3}$, 10$^{-4}$ cm$^{-3}$, 10$^{-6}$ cm$^{-3}$), see Figure 1.



\begin{figure}[h!]
  \centering
  \includegraphics[width=0.9\linewidth]{./plots/cooling1a.png}
  \caption{The result of our program shows the total cooling rate as a function of the temperature at z=3.
   It roughly decreases with the Hydrogen density and increases with temperature. The result is similar to Figure 2 in the referenced literature(Wiersma et al.2009).}
  \label{fig1}
\end{figure}

\subsection{b}
Now, we set the metallicity to be 0.5, density to 0.0001 cm$^{-3}$ and calculate the cooling rate as a function of temperature for the allowed redshift (0-8.989) range.
We need to make a movie to show the variations with redshift. We used 100 different redshifts and a framerate of 10 in this movie.\\

The code specific to this question is give by:
\lstinputlisting[firstline=137, lastline=158]{cooling.py}

For the movie see the main directory the file \texttt{cooling\_ratemovie.mp4}. We can see that the cooling rate increases with redshift.
